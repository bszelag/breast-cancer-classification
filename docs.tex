%Zawartość dokumentacji: 
%- Wstęp - zawierający cel i zakres projektu, uzasadnienie biznesowe, słownik pojęć
%- Analizę systemu - krótki opis działania systemu, wymagania funkcjonalne i niefunkcjonalne
%- Projekt systemu - opis wykorzystywanych technologii, diagramy klas, schemat systemu, opis działania algorytmów itp.
%- Opis techniczny - opis implementacji poszczegółnych elementów, istotne fragmenty kodu
%- Opis funkcjonalny - instalacja i konfiguracja systemu, opis wszystkich dostępnych funkcjonalności
%- Testy wydajnościowe - opis przeprowadzonych testów, zestawienie wyników, analiza wyników
%- Spisy tabel, obrazów 



\documentclass[a4paper, 11pt]{report}
\usepackage{listings}
\usepackage[hyphens]{url}
\usepackage{rotating}
\usepackage{tikz}
\usepackage{float}
\usepackage[T1]{fontenc}
\usepackage[polish]{babel}
\usepackage[utf8]{inputenc}
\usepackage[export]{adjustbox}
\usepackage{lmodern}
\usepackage{indentfirst}
\usepackage{multirow}
\usepackage{longtable}
\usepackage{geometry}
\usepackage[section]{placeins}
\setlength{\parskip}{1em}
\usepackage{enumerate}
\usepackage[noend]{algorithmic}

\usepackage{listings}
\usepackage{color}
\definecolor{pblue}{rgb}{0.13,0.13,1}
\definecolor{pgreen}{rgb}{0,0.5,0}
\definecolor{pred}{rgb}{0.9,0,0}
\definecolor{pgrey}{rgb}{0.46,0.45,0.48}

\usepackage{listings}

\lstdefinelanguage{JavaScript}{
  keywords={break, case, catch, continue, debugger, default, delete, do, else, finally, for, function, if, in, instanceof, new, return, switch, this, throw, try, typeof, var, void, while, with},
  morecomment=[l]{//},
  morecomment=[s]{/*}{*/},
  morestring=[b]',
  morestring=[b]",
  sensitive=true
}

\lstset{
  showspaces=false,
  showtabs=false,
  breaklines=true,
  showstringspaces=false,
  breakatwhitespace=true,
  commentstyle=\color{pgreen},
  keywordstyle=\color{pblue},
  stringstyle=\color{pred},
  basicstyle=\ttfamily,
  moredelim=[il][\textcolor{pgrey}]
  moredelim=[is][\textcolor{pgrey}]{\%\%}{\%\%}
}
\selectlanguage{polish}
\author{Mateusz~Gniewkowski 218138 \\
		Beata~Szeląg 218139}
\title{\huge Klasyfikacja danych dt. raka piersi \\
			 Informatyka w medycynie - projekt}

\date {}
\usepackage{graphicx} 

\renewcommand{\lstlistingname}{Code}
\lstset{
    frame=tb, %%tb
    tabsize=4, % tab space width
    showstringspaces=false, % don't mark spaces in strings
    keywordstyle=\color{blue}, % keyword color
    numbers=none,
}
\begin{document}
\maketitle
\tableofcontents{}


%%%%%%%%%%%%%%%%%%%%%%%%%%%%%%%%%%%%%%%%%%%%%%%%%%%%%%%%%%%%%%%%%%%%%% 
 
 
\chapter{Wstęp}


Celem projektu jest ....

W ramach projektu zostanie zaprojektowany i zaimplementowany ...

\section{Uzasadnienie biznesowe}

...

\chapter{Analiza systemu}

	\section{Opis działania systemu}
	
	System pozwala na ...
		
	\section{Wymagania funkcjonalne}
	
	\section{Wymagania niefunkcjonalne}
	
	
\chapter{Projekt systemu}

	\section{Backend}
	
	\subsection{Klasyfikacja}	
	
	\subsection{REST API}
	
	
	\section{Aplikacja internetowa}
	
Aplikacja kliencka ma umożliwiać użytkownikom wygodny dostęp do oferowanych przez nasz system funkcjonalności. 	Składa się ona z dwóch głównych części:
	
	\begin{enumerate}
	\item menu w górnej części ekranu
	\item głównej części ekranu, w której wyświetlamy poszczególne widoki.
	\end{enumerate}
	
\noindent
Widoki aplikacji:
\begin{itemize}
\item hghgfgh
\end{itemize}

\noindent
Do stworzenia aplikacji klienckiej wykorzystaliśmy Vue.js oraz dodatkowo:
\begin{itemize}
\item bootstrap
\item font awsome 
\item axios
\end{itemize}

\noindent
Działanie aplikacji internetowej:
\begin{itemize}
\item przejścia pomiędzy kolejnymi widokami bjjkhjkhkjhjkhjkhjkh
\item aplikacja komunikuje się z serwerem poprzez REST API wysyłając zapytania http na zdefiniowane endpointy
\end{itemize}

	
	\section{Baza danych}
	
W naszym projekcie postanowiliśmy skorzystać z nierelacjnej bazy danych NoSql jaką jest MongoDB. MongoDB jest bazą przechowującą dokumenty w formacie json. Bazy danych nie dzieli się na tabelę a na kolekcje, które przechowują dany typ dokumentów reprezentujących jakiś obiekt. Zaplanowaliśmy stworzenie dwóch kolekcji - ghhghghfhg...........hhghghghghg
		
\chapter{Opis techniczny}

	\section{Backend}
	
	\section{Aplikacja internetowa}

		\subsection{Szablon aplikacji internetowej}
		
		\subsection{Trenowanie}
		\subsection{Predykcja}
		\subsection{Statystyki}
	
	\section{Baza danych}


\chapter{Opis funkcjonalny}

	\section{Instalacja i konfiguracja systemu}
	

\chapter{Testy}

	Testy zostały przeprowadzone kjhkjhjkhjhgjhjhhjjhhjgjhg
	
	\section{Opis i zestawienie wyników}
	
	Zebrane wyniki zostały zestawione i przedstawione w poniższej tabeli oraz na wykresie.
	
%	\begin{longtable}{|c|c|c|}
%		\caption{Tabela zależności czasu wykonania algorytmu od ilości slaveów dla poszczególnych instancji} \\ \hline
%		\multirow{2}{*}{slaves}  & \multicolumn{2}{ |c| }{ Czas [s] } \\  \cline{2-3}
%		& MIN & MID \\ \hline
%		1 & 36 & 78 \\ \hline
%		2 & 30 & 45 \\ \hline
%		4 & 36 & 40 \\ \hline
%	\end{longtable}

%	\begin{figure}[!ht]
%		\centering
%		\caption{Wykres zależności czasu wykonania algorytmu od ilości slaveów dla poszczególnych instancji} 
%		\includegraphics[width=5in]{WykresCzasOdSlave}
%		\label{fig_diag}
%	\end{figure}
%	
	
	\section{Analiza wyników}	

\chapter{Spis tabel i obrazów}


\begingroup
\let\clearpage\relax
\listoffigures
\listoftables
\endgroup


\end{document}