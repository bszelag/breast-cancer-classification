%Zawartość dokumentacji: 
%- Wstęp - zawierający cel i zakres projektu, uzasadnienie biznesowe, słownik pojęć
%- Analizę systemu - krótki opis działania systemu, wymagania funkcjonalne i niefunkcjonalne
%- Projekt systemu - opis wykorzystywanych technologii, diagramy klas, schemat systemu, opis działania algorytmów itp.
%- Opis techniczny - opis implementacji poszczegółnych elementów, istotne fragmenty kodu
%- Opis funkcjonalny - instalacja i konfiguracja systemu, opis wszystkich dostępnych funkcjonalności
%- Testy wydajnościowe - opis przeprowadzonych testów, zestawienie wyników, analiza wyników
%- Spisy tabel, obrazów 



\documentclass[a4paper, 11pt]{report}
\usepackage{listings}
\usepackage[hyphens]{url}
\usepackage{rotating}
\usepackage{tikz}
\usepackage{float}
\usepackage[T1]{fontenc}
\usepackage[polish]{babel}
\usepackage[utf8]{inputenc}
\usepackage[export]{adjustbox}
\usepackage{lmodern}
\usepackage{indentfirst}
\usepackage{multirow}
\usepackage{longtable}
\usepackage{geometry}
\usepackage[section]{placeins}
\setlength{\parskip}{1em}
\usepackage{enumerate}
\usepackage[noend]{algorithmic}

\usepackage{listings}
\usepackage{color}
\definecolor{pblue}{rgb}{0.13,0.13,1}
\definecolor{pgreen}{rgb}{0,0.5,0}
\definecolor{pred}{rgb}{0.9,0,0}
\definecolor{pgrey}{rgb}{0.46,0.45,0.48}

\usepackage{listings}

\lstdefinelanguage{JavaScript}{
  keywords={break, case, catch, continue, debugger, default, delete, do, else, finally, for, function, if, in, instanceof, new, return, switch, this, throw, try, typeof, var, void, while, with},
  morecomment=[l]{//},
  morecomment=[s]{/*}{*/},
  morestring=[b]',
  morestring=[b]",
  sensitive=true
}

\lstset{
  showspaces=false,
  showtabs=false,
  breaklines=true,
  showstringspaces=false,
  breakatwhitespace=true,
  commentstyle=\color{pgreen},
  keywordstyle=\color{pblue},
  stringstyle=\color{pred},
  basicstyle=\ttfamily,
  moredelim=[il][\textcolor{pgrey}]
  moredelim=[is][\textcolor{pgrey}]{\%\%}{\%\%}
}
\selectlanguage{polish}
\author{Mateusz~Gniewkowski 218138 \\
		Beata~Szeląg 218139}
\title{\huge Klasyfikacja danych dt. raka piersi \\
			 Informatyka w medycynie - projekt}

\date {}
\usepackage{graphicx} 

\renewcommand{\lstlistingname}{Code}
\lstset{
    frame=tb, %%tb
    tabsize=4, % tab space width
    showstringspaces=false, % don't mark spaces in strings
    keywordstyle=\color{blue}, % keyword color
    numbers=none,
}
\begin{document}
\maketitle
\tableofcontents{}


%%%%%%%%%%%%%%%%%%%%%%%%%%%%%%%%%%%%%%%%%%%%%%%%%%%%%%%%%%%%%%%%%%%%%% 
 
 
\chapter{Wstęp}


Celem projektu jest zaprojektowanie systemu umożliwiającego klasyfikację danych medycznych. Wykorzystywane są ogólnodostepne dane dotyczące raka piersi pochodzące ze strony \textit{https://archive.ics.uci.edu/ml/machine-learning-databases/breast-cancer-wisconsin}.

Każda próbka obecna w zbiorze posiada swój numer id oraz 9 cech, które ją opisują. Są to:
\begin{enumerate}
\item Clump Thickness              
\item Uniformity of Cell Size       
\item Uniformity of Cell Shape     
\item Marginal Adhesion             
\item Single Epithelial Cell Size  
\item Bare Nuclei                  
\item Bland Chromatin              
\item Normal Nucleoli               
\item Mitoses                       
\end{enumerate}
Cechy te zostały określone na podstawie biopsji. Ich wartości znajdują się w zakresie 1-10.
Klasy, które są rozpatrywane w ramach tego zbioru dsanych to:
\begin{itemize}
\item benign - nowotwór łagodny
\item malignant - nowotwór złośliwy
\end{itemize}

W ramach projektu zostanie zaprojektowany i zaimplementowany system pozwalający na tworzenie i trenowanie różnego rodzaju klasyfikatorów, następnie wykorzystanie ich do przydzielenia odpowiedniej klasy nowym danym. Ponadto możliwe będzie przeglądanie historii klasyfikatorów.


\chapter{Analiza systemu}

	\section{Opis działania systemu}
	
	System pozwala na stworzenie i trening klasyfikatora bazując na danych wejściowych dostarczanych przez użytkownika. Dane są wprowadzane w formie pliku csv. Użytkownik może wybrać rodzaj klasyfikatora oraz może okreslić charakterystyczne dla niego opcje. Po stworzeniu klasyfikatora można go użyć do predykcji - w tym celu użytkownik powinien podać dane w formacie csv. Istnieje możliwość sprawdzenia skuteczności aktualnego klasyfikatora - w tym celu należy wprowadzić dane wraz z określonymi klasami.
	
	\subsection{Użytkownik}
	Użytkownik systemu nie musi mieć wiedzy na temat klasyfikatorów, czy ich działania. Jednak będzie to pomocne w uzyskaniu lepszych wyników (pomoże to wybrać najlepsze opcje dotyczące klasyfikatora).
		
	\section{Wymagania funkcjonalne}
	\begin{enumerate}
	\item Użytkownik ma możliwość stworzenia (wytrenowania) klasyfikatora, w tym celu powinien podać:
		\begin{itemize}
		\item plik z danymi w formacie csv 
		\item wybrać rodzaj klasyfikatora (Naive Bayes, SVM, drzewo decyzyjne)
		\item wybrać opcje charakterystyczne dla danego klasyfikatora:
			\begin{itemize}
			\item Naive Bayes: rodzaj (dystrybucja)
			\item SVM: kernel, C, gamma
			\item drzewo decyzyjne: criterion, splitter, minimal split of samples
			\end{itemize}
		\item opcjonalnie: określić, czy klasyfikator powinien korzystać z selekcji cech
		\end{itemize}
	\end{enumerate}
	
	\section{Wymagania niefunkcjonalne}
	
	\begin{enumerate}
	\item Aplikacja powinna zapewniać prosty i przejrzysty interfejs graficzny użytkownika.
	\item System powinien zapewnić możliwość rozbudowy (łatwe dodawanie nowych klasyfikatorów, ich opcji)
	\item System powinien działać na przeglądarkach Google Chrome (od wersji 49), Mozilla Firefox (od wersji 59), Safari (od wersji 11), Edge (od wersji 16) oraz na mobilnych przeglądarkach Opera Mini i Chrome w wersji na Androida (od wersji 66)
	\end{enumerate}
	
	
\chapter{Projekt systemu}

	\section{Backend}
	
		\subsection{Klasyfikacja}	
	
		\subsection{REST API}
	
	
	\section{Aplikacja internetowa}
	
Aplikacja kliencka ma umożliwiać użytkownikom wygodny dostęp do oferowanych przez nasz system funkcjonalności. 	Składa się ona z dwóch głównych części:
	
	\begin{enumerate}
	\item menu w górnej części ekranu
	\item głównej części ekranu, w której wyświetlamy poszczególne widoki.
	\end{enumerate}
	
\noindent
Widoki aplikacji:
\begin{itemize}
\item trening
\item predykcja
\item statystyki
\end{itemize}

\noindent
Do stworzenia aplikacji klienckiej wykorzystaliśmy Vue.js w wersji 2 oraz dodatkowo:
\begin{itemize}
\item bootstrap - biblioteka frontendowa
\item font awsome - zbiór ikon
\item axios - bilbioteka pozwalająca na tworzenie i wysyłanie zapytań REST
\end{itemize}

\noindent
Działanie aplikacji internetowej:
\begin{itemize}
\item przejścia pomiędzy kolejnymi widokami (trenowanie - predykcja - statystyki)
\item aplikacja komunikuje się z serwerem poprzez REST API wysyłając zapytania http na zdefiniowane endpointy
\end{itemize}

	
	\section{Baza danych}
	
W naszym projekcie postanowiliśmy skorzystać z nierelacjnej bazy danych NoSql jaką jest MongoDB. MongoDB jest bazą przechowującą dokumenty w formacie json. Bazy danych nie dzieli się na tabelę a na kolekcje, które przechowują dany typ dokumentów reprezentujących jakiś obiekt. Zaplanowaliśmy stworzenie dwóch kolekcji - ...
		
\chapter{Opis techniczny}

	\section{Backend}
	
	\section{Aplikacja internetowa}

		\subsection{Szablon aplikacji internetowej}
		\begin{itemize}
		\item aplikacja jest podzielona na komponenty
		\item dany komponent ma określone zadanie, np: komponent Results zawiera dane dotyczące danej predykcji, pozwala na ich przetworzenie i wyświetlenie
		\end{itemize}
		
		\subsection{Trenowanie}
		Komponent ten składa się z formularza, zawierającego pola potrzebne do stworzenia, treningu klasyfikatora. W zależności od wybranego algorytmu, wyświetla konkretne opcje do wybrania. Ponadto w ramach tego komponentu przygotowywane jest zapytanie do serwera, które umożliwia przesłanie danych z formularza. 
		
		\subsection{Predykcja}
		Konponent sklada się z dwóch części: formularza i wyników predykcji. W formularzu użytkownik może okreslić rodzaj klasyfikatora oraz to, czy dane zawierają już klasy. W tej części przygotowywane jest również zapytanie wysyłające opisane dane do serwera. Następnie w ramach tego komponentu można wyświetlić wyniki predykcji. 
		
		\subsection{Wyświetlanie wyników}
		\begin{itemize}
		\item macierz błedu w formie wykresu kołowego
		\item dodatkowe metryki określające trafność: (ERR, ACC, SN, SP, PREC, FPR, F1, MCC)
		\item czas treningu
		\item rozmiar pliku treningowego
		\item wyniki klasyfikacji próbek
		\end{itemize}
		
		\subsection{Statystyki}
		Komponent zawiera wyniki wszystkich przeprowadzonych predykcji, zebrane w formie tabeli. Ponadto pozwala na sortowanie wyników według rodzaju algorytmu, rozmiaru pliku treningowego, liczby próbek w pliku, czasu treningu.
		Dodatkowo wyświetlane są macierze błędu (w formie wykresów kołowych) dla każdego z trzech dostepnych algorytmów: Naive Bayes, SVM, drzewa decyzyjnego.
	
	\section{Baza danych}


\chapter{Opis funkcjonalny}

	\section{Instalacja i konfiguracja systemu}
	

\chapter{Testy}

	Testy zostały przeprowadzone kjhkjhjkhjhgjhjhhjjhhjgjhg
	
	\section{Opis i zestawienie wyników}
	
	Zebrane wyniki zostały zestawione i przedstawione w poniższej tabeli oraz na wykresie.
	
%	\begin{longtable}{|c|c|c|}
%		\caption{Tabela zależności czasu wykonania algorytmu od ilości slaveów dla poszczególnych instancji} \\ \hline
%		\multirow{2}{*}{slaves}  & \multicolumn{2}{ |c| }{ Czas [s] } \\  \cline{2-3}
%		& MIN & MID \\ \hline
%		1 & 36 & 78 \\ \hline
%		2 & 30 & 45 \\ \hline
%		4 & 36 & 40 \\ \hline
%	\end{longtable}

%	\begin{figure}[!ht]
%		\centering
%		\caption{Wykres zależności czasu wykonania algorytmu od ilości slaveów dla poszczególnych instancji} 
%		\includegraphics[width=5in]{WykresCzasOdSlave}
%		\label{fig_diag}
%	\end{figure}
%	
	
	\section{Analiza wyników}	

\chapter{Spis tabel i obrazów}


\begingroup
\let\clearpage\relax
\listoffigures
\listoftables
\endgroup


\end{document}